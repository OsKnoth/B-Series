\documentclass[english,unicode]{article}
%\documentclass[11pt]{beamer}
%\usetheme{AnnArbor}
%\usetheme{CambridgeUS}
%\usetheme{Pittsburgh}
%\usetheme{Madrid}
%\usetheme{Antibes}
%\usetheme{}
\usepackage[utf8]{inputenc}
\usepackage[german]{babel}
\usepackage[T1]{fontenc}
\usepackage{amsmath}
\usepackage{amsfonts}
\usepackage{amssymb}
\usepackage{graphicx}
\usepackage{tikz}
\usepackage{forest}
\usepackage{algorithm}
\usepackage{algorithmic}
\usepackage{lscape}
\author{Oswald Knoth, J\"org Wensch}
\title{B-Series and Split-Explicit Methods}
%\setbeamercovered{transparent} 
%\setbeamertemplate{navigation symbols}{} 
%\logo{} 

 
\usepackage{subfigure}
\newtheorem{theorem}{Theorem}
\def\treea{
\begin{forest} 
[.,node options ={circle,fill=black,draw=black,scale=0.3}]
\end{forest}}

\def\treeaw{
\begin{forest} 
[.,node options ={circle,fill=white,draw=black,scale=0.3}]
\end{forest}}

\def\treeb{
\begin{forest}
[.,node options ={circle,fill=black,draw=black,scale=0.3},
for tree={grow=90,s sep=1.5pt,l=1pt}
[.,node options ={circle,fill=black,draw=black,scale=0.3}]]
\end{forest}}

\def\treew{
\begin{forest}
[.,node options ={circle,fill=black,draw=black,scale=0.3},
for tree={grow=90,s sep=1.5pt,l=1pt}
[.,node options ={circle,fill=white,draw=black,scale=0.3}]]
\end{forest}}

\def\treebwb{
\begin{forest}
[.,node options ={circle,fill=white,draw=black,scale=0.3},
for tree={grow=90,s sep=1.5pt,l=1pt}
[.,node options ={circle,fill=black,draw=black,scale=0.3}]]
\end{forest}}

\def\treebbw{
\begin{forest}
[.,node options ={circle,fill=black,draw=black,scale=0.3},
for tree={grow=90,s sep=1.5pt,l=1pt}
[.,node options ={circle,fill=white,draw=black,scale=0.3}]]
\end{forest}}

\def\treebww{
\begin{forest}
[.,node options ={circle,fill=white,draw=black,scale=0.3},
for tree={grow=90,s sep=1.5pt,l=1pt}
[.,node options ={circle,fill=white,draw=black,scale=0.3}]]
\end{forest}}

\def\treec{
\begin{forest}
[.,node options ={circle,fill=black,draw=black,scale=0.3},
for tree={grow=90,s sep=1.5pt,l=1pt}
[.,node options ={circle,fill=black,draw=black,scale=0.3}]
[.,node options ={circle,fill=black,draw=black,scale=0.3}]]
\end{forest}}

\def\treecc{
\begin{forest}
[.,node options ={circle,fill=black,draw=black,scale=0.3},
for tree={grow=90,s sep=1.5pt,l=1pt}
[.,node options ={circle,fill=black,draw=black,scale=0.3}]
[.,node options ={circle,fill=white,draw=black,scale=0.3}]]
\end{forest}}

\def\treeccc{
\begin{forest}
[.,node options ={circle,fill=black,draw=black,scale=0.3},
for tree={grow=90,s sep=1.5pt,l=1pt}
[.,node options ={circle,fill=white,draw=black,scale=0.3}]
[.,node options ={circle,fill=white,draw=black,scale=0.3}]]
\end{forest}}

\def\treecwbb{
\begin{forest}
[.,node options ={circle,fill=white,draw=black,scale=0.3},
for tree={grow=90,s sep=1.5pt,l=1pt}
[.,node options ={circle,fill=black,draw=black,scale=0.3}]
[.,node options ={circle,fill=black,draw=black,scale=0.3}]]
\end{forest}}

\def\treed{
\begin{forest}
[.,node options ={circle,fill=black,draw=black,scale=0.3},
for tree={grow=90,s sep=1.5pt,l=1pt},
[,node options ={circle,fill=black,draw=black,scale=0.3}
[.,node options ={circle,fill=black,draw=black,scale=0.3}]]]
\end{forest}}

\def\treedd{
\begin{forest}
[.,node options ={circle,fill=black,draw=black,scale=0.3},
for tree={grow=90,s sep=1.5pt,l=1pt},
[,node options ={circle,fill=black,draw=black,scale=0.3}
[.,node options ={circle,fill=white,draw=black,scale=0.3}]]]
\end{forest}}

\def\treeddd{
\begin{forest}
[.,node options ={circle,fill=black,draw=black,scale=0.3},
for tree={grow=90,s sep=1.5pt,l=1pt},
[,node options ={circle,fill=white,draw=black,scale=0.3}
[.,node options ={circle,fill=black,draw=black,scale=0.3}]]]
\end{forest}}

\def\treedddd{
\begin{forest}
[.,node options ={circle,fill=black,draw=black,scale=0.3},
for tree={grow=90,s sep=1.5pt,l=1pt},
[,node options ={circle,fill=white,draw=black,scale=0.3}
[.,node options ={circle,fill=white,draw=black,scale=0.3}]]]
\end{forest}}

\def\treef{
\begin{forest}
[.,node options ={circle,fill=black,draw=black,scale=0.3},
for tree={grow=90,s sep=1.5pt,l=1pt}
[.,node options ={circle,fill=black,draw=black,scale=0.3}]
[.,node options ={circle,fill=black,draw=black,scale=0.3}]
[.,node options ={circle,fill=black,draw=black,scale=0.3}]]
\end{forest}}

\def\treeff{
\begin{forest}
[.,node options ={circle,fill=black,draw=black,scale=0.3},
for tree={grow=90,s sep=1.5pt,l=1pt}
[.,node options ={circle,fill=white,draw=black,scale=0.3}]
[.,node options ={circle,fill=black,draw=black,scale=0.3}]
[.,node options ={circle,fill=black,draw=black,scale=0.3}]]
\end{forest}}

\def\treefff{
\begin{forest}
[.,node options ={circle,fill=black,draw=black,scale=0.3},
for tree={grow=90,s sep=1.5pt,l=1pt}
[.,node options ={circle,fill=white,draw=black,scale=0.3}]
[.,node options ={circle,fill=white,draw=black,scale=0.3}]
[.,node options ={circle,fill=black,draw=black,scale=0.3}]]
\end{forest}}

\def\treeffff{
\begin{forest}
[.,node options ={circle,fill=black,draw=black,scale=0.3},
for tree={grow=90,s sep=1.5pt,l=1pt}
[.,node options ={circle,fill=white,draw=black,scale=0.3}]
[.,node options ={circle,fill=white,draw=black,scale=0.3}]
[.,node options ={circle,fill=white,draw=black,scale=0.3}]]
\end{forest}}

\def\treeg{
\begin{forest}
[.,node options ={circle,fill=black,draw=black,scale=0.3},
for tree={grow=90,s sep=1.5pt,l=1pt},
[,node options ={circle,fill=black,draw=black,scale=0.3}
[.,node options ={circle,fill=black,draw=black,scale=0.3}]]
[.,node options ={circle,fill=black,draw=black,scale=0.3}]]
\end{forest}}

\def\treegg{
\begin{forest}
[.,node options ={circle,fill=black,draw=black,scale=0.3},
for tree={grow=90,s sep=1.5pt,l=1pt},
[,node options ={circle,fill=white,draw=black,scale=0.3}
[.,node options ={circle,fill=black,draw=black,scale=0.3}]]
[.,node options ={circle,fill=black,draw=black,scale=0.3}]]
\end{forest}}

\def\treeggg{
\begin{forest}
[.,node options ={circle,fill=black,draw=black,scale=0.3},
for tree={grow=90,s sep=1.5pt,l=1pt},
[,node options ={circle,fill=black,draw=black,scale=0.3}
[.,node options ={circle,fill=white,draw=black,scale=0.3}]]
[.,node options ={circle,fill=black,draw=black,scale=0.3}]]
\end{forest}}

\def\treegggg{
\begin{forest}
[.,node options ={circle,fill=black,draw=black,scale=0.3},
for tree={grow=90,s sep=1.5pt,l=1pt},
[,node options ={circle,fill=black,draw=black,scale=0.3}
[.,node options ={circle,fill=black,draw=black,scale=0.3}]]
[.,node options ={circle,fill=white,draw=black,scale=0.3}]]
\end{forest}}


\def\treeggG{
\begin{forest}
[.,node options ={circle,fill=black,draw=black,scale=0.3},
for tree={grow=90,s sep=1.5pt,l=1pt},
[,node options ={circle,fill=white,draw=black,scale=0.3}
[.,node options ={circle,fill=white,draw=black,scale=0.3}]]
[.,node options ={circle,fill=black,draw=black,scale=0.3}]]
\end{forest}}

\def\treeggGg{
\begin{forest}
[.,node options ={circle,fill=black,draw=black,scale=0.3},
for tree={grow=90,s sep=1.5pt,l=1pt},
[,node options ={circle,fill=black,draw=black,scale=0.3}
[.,node options ={circle,fill=white,draw=black,scale=0.3}]]
[.,node options ={circle,fill=white,draw=black,scale=0.3}]]
\end{forest}}

\def\treegGgg{
\begin{forest}
[.,node options ={circle,fill=black,draw=black,scale=0.3},
for tree={grow=90,s sep=1.5pt,l=1pt},
[,node options ={circle,fill=white,draw=black,scale=0.3}
[.,node options ={circle,fill=black,draw=black,scale=0.3}]]
[.,node options ={circle,fill=white,draw=black,scale=0.3}]]
\end{forest}}

\def\treeG{
\begin{forest}
[.,node options ={circle,fill=black,draw=black,scale=0.3},
for tree={grow=90,s sep=1.5pt,l=1pt},
[,node options ={circle,fill=white,draw=black,scale=0.3}
[.,node options ={circle,fill=white,draw=black,scale=0.3}]]
[.,node options ={circle,fill=white,draw=black,scale=0.3}]]
\end{forest}}

\def\treeh{
\begin{forest}
[.,node options ={circle,fill=black,draw=black,scale=0.3},
for tree={grow=90,s sep=1.5pt,l=1pt}
[.,node options ={circle,fill=black,draw=black,scale=0.3}
[.,node options ={circle,fill=black,draw=black,scale=0.3}]
[.,node options ={circle,fill=black,draw=black,scale=0.3}]]]
\end{forest}}

\def\treehh{
\begin{forest}
[.,node options ={circle,fill=black,draw=black,scale=0.3},
for tree={grow=90,s sep=1.5pt,l=1pt}
[.,node options ={circle,fill=white,draw=black,scale=0.3}
[.,node options ={circle,fill=black,draw=black,scale=0.3}]
[.,node options ={circle,fill=black,draw=black,scale=0.3}]]]
\end{forest}}

\def\treehhh{
\begin{forest}
[.,node options ={circle,fill=black,draw=black,scale=0.3},
for tree={grow=90,s sep=1.5pt,l=1pt}
[.,node options ={circle,fill=black,draw=black,scale=0.3}
[.,node options ={circle,fill=white,draw=black,scale=0.3}]
[.,node options ={circle,fill=black,draw=black,scale=0.3}]]]
\end{forest}}

\def\treehhhh{
\begin{forest}
[.,node options ={circle,fill=black,draw=black,scale=0.3},
for tree={grow=90,s sep=1.5pt,l=1pt}
[.,node options ={circle,fill=white,draw=black,scale=0.3}
[.,node options ={circle,fill=white,draw=black,scale=0.3}]
[.,node options ={circle,fill=black,draw=black,scale=0.3}]]]
\end{forest}}

\def\treehhhhh{
\begin{forest}
[.,node options ={circle,fill=black,draw=black,scale=0.3},
for tree={grow=90,s sep=1.5pt,l=1pt}
[.,node options ={circle,fill=white,draw=black,scale=0.3}
[.,node options ={circle,fill=white,draw=black,scale=0.3}]
[.,node options ={circle,fill=white,draw=black,scale=0.3}]]]
\end{forest}}

\def\treehhhhhh{
\begin{forest}
[.,node options ={circle,fill=black,draw=black,scale=0.3},
for tree={grow=90,s sep=1.5pt,l=1pt}
[.,node options ={circle,fill=black,draw=black,scale=0.3}
[.,node options ={circle,fill=white,draw=black,scale=0.3}]
[.,node options ={circle,fill=white,draw=black,scale=0.3}]]]
\end{forest}}


\def\treei{
\begin{forest}
[.,node options ={circle,fill=black,draw=black,scale=0.3},
for tree={grow=90,s sep=1.5pt,l=1pt},
[,node options ={circle,fill=black,draw=black,scale=0.3}
[.,node options ={circle,fill=black,draw=black,scale=0.3}
[.,node options ={circle,fill=black,draw=black,scale=0.3}]]]]
\end{forest}}


\def\treeii{
\begin{forest}
[.,node options ={circle,fill=black,draw=black,scale=0.3},
for tree={grow=90,s sep=1.5pt,l=1pt},
[,node options ={circle,fill=white,draw=black,scale=0.3}
[.,node options ={circle,fill=black,draw=black,scale=0.3}
[.,node options ={circle,fill=black,draw=black,scale=0.3}]]]]
\end{forest}}

\def\treeiii{
\begin{forest}
[.,node options ={circle,fill=black,draw=black,scale=0.3},
for tree={grow=90,s sep=1.5pt,l=1pt},
[,node options ={circle,fill=black,draw=black,scale=0.3}
[.,node options ={circle,fill=white,draw=black,scale=0.3}
[.,node options ={circle,fill=black,draw=black,scale=0.3}]]]]
\end{forest}}


\def\treeiiii{
\begin{forest}
[.,node options ={circle,fill=black,draw=black,scale=0.3},
for tree={grow=90,s sep=1.5pt,l=1pt},
[,node options ={circle,fill=black,draw=black,scale=0.3}
[.,node options ={circle,fill=black,draw=black,scale=0.3}
[.,node options ={circle,fill=white,draw=black,scale=0.3}]]]]
\end{forest}}

\def\treeI{
\begin{forest}
[.,node options ={circle,fill=black,draw=black,scale=0.3},
for tree={grow=90,s sep=1.5pt,l=1pt},
[,node options ={circle,fill=white,draw=black,scale=0.3}
[.,node options ={circle,fill=white,draw=black,scale=0.3}
[.,node options ={circle,fill=white,draw=black,scale=0.3}]]]]
\end{forest}}

\def\treeII{
\begin{forest}
[.,node options ={circle,fill=black,draw=black,scale=0.3},
for tree={grow=90,s sep=1.5pt,l=1pt},
[,node options ={circle,fill=black,draw=black,scale=0.3}
[.,node options ={circle,fill=white,draw=black,scale=0.3}
[.,node options ={circle,fill=white,draw=black,scale=0.3}]]]]
\end{forest}}

\def\treeIII{
\begin{forest}
[.,node options ={circle,fill=black,draw=black,scale=0.3},
for tree={grow=90,s sep=1.5pt,l=1pt},
[,node options ={circle,fill=white,draw=black,scale=0.3}
[.,node options ={circle,fill=black,draw=black,scale=0.3}
[.,node options ={circle,fill=white,draw=black,scale=0.3}]]]]
\end{forest}}

\def\treeIIII{
\begin{forest}
[.,node options ={circle,fill=black,draw=black,scale=0.3},
for tree={grow=90,s sep=1.5pt,l=1pt},
[,node options ={circle,fill=white,draw=black,scale=0.3}
[.,node options ={circle,fill=white,draw=black,scale=0.3}
[.,node options ={circle,fill=black,draw=black,scale=0.3}]]]]
\end{forest}}

%\include{trees}
\def\Ri{\sum_{j=1}^{i-1}r_{ij}}
\def\Rj{\sum_{k=1}^{j-1}r_{jk}}
\def\Rk{\sum_{l=1}^{k-1}r_{kl}}
\def\Rl{\sum_{r=1}^{l-1}r_{lr}}

\def\Di{\sum_{j=1}^{i-1}d_{ij}}
\def\Dj{\sum_{k=1}^{j-1}d_{jk}}
\def\Dk{\sum_{l=1}^{k-1}d_{kl}}
\def\Dl{\sum_{r=1}^{l-1}d_{lr}}

\def\SiL{\sum_{j=1}^{i-1}a_{ij}^{[0]}[\lambda]}
\def\SjL{\sum_{k=1}^{j-1}a_{jk}^{[0]}[\lambda]}
\def\SiIL{\sum_{j=1}^{i-1}a_{ij}^{[1]}[\lambda]}
\def\SjIL{\sum_{k=1}^{j-1}a_{jk}^{[1]}[\lambda]}
\def\Si{\sum_{j=1}^{i-1}a_{ij}^{[0]}}
\def\Sj{\sum_{k=1}^{j-1}a_{jk}^{[0]}}
\def\Sk{\sum_{l=1}^{k-1}a_{kl}^{[0]}}
\def\Sl{\sum_{r=1}^{l-1}a_{lr}^{[0]}}


\def\SiI{\sum_{j=1}^{i-1}a_{ij}^{[1]}}
\def\SjI{\sum_{k=1}^{j-1}a_{jk}^{[1]}}
\def\SkI{\sum_{l=1}^{k-1}a_{kl}^{[1]}}
\def\SlI{\sum_{r=1}^{l-1}a_{lr}^{[1]}}
\def\SiIO{\sum_{j=1}^{i-1}a_{ij}^{[1]}[1]}
\def\SjIO{\sum_{k=1}^{j-1}a_{jk}^{[1]}[1]}
\def\SkIO{\sum_{l=1}^{k-1}a_{kl}^{[1]}[1]}
\def\SlIO{\sum_{r=1}^{l-1}a_{lr}^{[1]}[1]}
\def\SrIO{\sum_{s=1}^{r-1}a_{rs}^{[1]}[1]}


\def\EIL{^{[1]}[\lambda]}
\def\EI{^{[1]}}
\def\EIO{^{[1]}[1]}
\def\L{[\lambda]}

\def\SSi{\sum a_{ij}^{[0]}}
\def\SSj{\sum a_{jk}^{[0]}}
\def\SSiI{\sum a_{ij}^{[1]}}
\def\SSjI{\sum a_{jk}^{[1]}}
\def\SSkI{\sum a_{kl}^{[1]}}
\def\SSiIO{\sum a_{ij}^{[1]}[1]}
\def\SSjIO{\sum a_{jk}^{[1]}[1]}
\def\SSkIO{\sum a_{kl}^{[1]}[1]}
\def\SSlIO{\sum a_{lr}^{[1]}[1]}


\newcommand{\pdn}[3]{\frac{\partial^{#3} #1}{{\partial #2}^{#3}}}
\newcommand{\pd}[2]{\frac{\partial #1}{\partial #2}}
\newcommand{\Ord}[1]{\mathcal{O}\left(#1\right)}
\newcommand{\real}{\mathbb{R}}
\newcommand{\realn}[1]{\mathbb{R}^{#1}}
\newcommand{\eps}{\varepsilon}
\newcommand{\eins}{\mbox{1\hskip-0.24em l}}
\newcommand{\id}{\operatorname{id}}
\newcommand{\follows}{\Rightarrow}
%\newcommand{\beins}{\mathbf{1}}
%\newcommand{\ol}[1]{\overline{#1}}
\newcommand{\half}{\frac{1}{2}}
\newtheorem{deff}{Definition}[section]
\newtheorem{satz}{Theorem}[section]

\begin{document}
\maketitle
%\newcommand[0]{\Si0L}{\sum_{j=1}^{i-1}a_\{ij}^{[0]}[\lambda]}
%\newcommand{\Si0L}{sum}


\section{Introduction}

MISB-Series is a FORTRAN program to compute order conditions for a class of one step methods where in each stage again a differential equation has to be solved. We start with the following differential equation

\begin{align*}
y' & = F(y,y) \hspace{3cm} 
\end{align*}
where the right hand side depends twice on the unknown $y$. This form of the right hand side involves a large number of ODE's for which special splitting typed integrator are proposed. We list types of equations which appear in the literature.

\noindent Other types of partitioning 

\noindent Additive splitting:
\begin{align*}
y' & = F(y,y) =f(y)+g(y)
\end{align*}
Additive nonlinear-linear splitting:
\begin{align*}
y' & = F(y,y) = f(y)+Ny
\end{align*}
Vector fields on manifolds in frame representation
\begin{align*}
y' &=F (y,y)=\sum_{i=1}^Nf_i(y)E_i(y)
\end{align*}
Multiplicative splitting:
\begin{align*}
y' &=F (y,y)=A(y)y
\end{align*}
I all examples the first $y$ argument is the first one in the general framework, and so on. 

For this type of equations exponential Runge-Kutta type methods will be analyzed. The methods have the following structure.

\begin{align*}
Y_1 & = y_n\\
Z'_i &=\sum_{j=1}^{i-1}\sum_{p=0}^{\rho_i}a_{ij}^p\left(\frac{\tau}{h}\right)^p F(Y_j,Z_i),\quad Z_i(0)=y_n+\sum_jd_{ij}(Y_j-y_n)\\
Y_i &  =Z_i(h),
\end{align*}
where in addition the consistency condition is required
\begin{align*}
\sum_{j=1}^{i-1} a_{ij}^p=0,\, p=1,\ldots,\rho_i.
\end{align*}

We will derive order conditions with the help of B-series. Let us start with the elementary differentials.
We will use the abbreviation $F^{(n)}_{(i,j)},\,i+j=n$ to denote the $i$-th derivative of f with respect to the first argument and the $j$-th derivative of f with respect to the second argument.

\begin{align*}
y' & = F\\
y'' & = F^{(1)}_{(1,0)}F+F^{(1)}_{(1,0)}F\\
y^{(3)}& = F^{(2)}_{(2,0)}FF+2F^{(2)}_{(1,1)}FF+F^{(2)}_{(0,2)}FF\\
& +F^{(1)}_{(1,0)}F^{(1)}_{(1,0)}F
+F^{(1)}_{(1,0)}F^{(1)}_{(0,1)}F
+F^{(1)}_{(0,1)}F^{(1)}_{(1,0)}F
+F^{(1)}_{(0,1)}F^{(1)}_{(0,1)}F
\end{align*}


\vspace{1cm}
For a graphical representation we use bi-coloured trees. Black vertices means computation of $F$ at the position
$(y,y)$ and and white vertices means computation at the position $(y,z)$. Upwards pointing branches represent
partial derivatives with respect to the first argument if the branch leads to a black vertex, and with respect to the second argument if it leads to a white vertex.
  
The trees are divided in two sets $T^b$ and $T^w$ where $T^b$ contain all trees with a black root and
 $T^w$ with a white root. Hence, the trees $T^b$ are computed at $(y,y)$ in the leading derivative whereas those of $T^w$ are computed at the mixed argument $(y,z)$. The tress  are defined recursively trough:
\begin{align*}
t & = [t_{b1}\ldots t_{bk},t_{w1}\ldots t_{wl}]_b, \quad \forall t \in T^b\\
t & = [t_{b1}\ldots t_{bk},t_{w1}\ldots t_{wl}]_w, \quad \forall t \in T^w
\end{align*}
Furthermore we introduce the black operator $^b$ which replaces a white root node by a black root node. 
\begin{align*}
t^b=[t_{b1}\ldots t_{bk},t_{w1}\ldots t_{wl}]_w^b=[t_{b1}\ldots t_{bk},t_{w1}\ldots t_{wl}]_b
\end{align*}
\begin{align*}
G_F( [t_{b,1}\ldots t_{b,k},t_{w1}\ldots t_{wl}]_b)(y_0,y_0) & =\\
F^{(k+l)}_{(k,l)}(y_0,y_0)(G(t_{b1})(y_0,y_0),\ldots,G(t_{bk})(y_0,y_0),G(t_{w1}^b)(y_0,y_0),\ldots,G(t_{wl}^b)(y_0,y_0))\\
G_F([t_{b,1}\ldots t_{b,k},t_{w1}\ldots t_{wl}]_w)(y_0,z_0) & = \\
F^{(k+l)}_{(k,l)}(y_0,y_0)(G(t_{b1})(y_0,y_0),\ldots,G(t_{bk})(y_0,y_0),G(t_{w1})(y_0,z_0),\ldots,G(t_{wl})(y_0,z_0))\\
\end{align*}
Two types of formal B-series are introduced
\begin{align*}
B(a,hF,y,y) & = a(\emptyset)y+\sum_{t\in T^b} \frac{a(t)}{\sigma(t)}G_F(t)(y,y)h^{r(t)}\\
C(b,hF,\lambda,y,z) & = b(\emptyset)z+\sum_{t\in T^w} \frac{b^{\lambda}(t)}{\sigma(t)}G_F(t)(y,z)h^{r(t)}
\end{align*}
where the coefficients $b^{\lambda}(t)$ of the second B-series  are polynomials in a further parameter $\lambda$
\begin{align*}
b^{\lambda}(t) & = b_0(t)+\lambda b_1(t)+\lambda^2 b_2(t)+\ldots
\end{align*}
%Development of $hF(y+B(a,hF,y,y) ,z+C(b,hF,\lambda,y,z))$ in a Taylor series yields
% \begin{align*}
% hF(y,z)+ h\sum_{m=1}^\infty \sum_{k=0}^m\frac{1}{m!}F^m_{(k,m-k)}(y,z)(& B(a,hF,y,y),\ldots,B(a,hF,y,y),\\ & C(b,hF,\lambda,y,z),\ldots,C(b,hF,\lambda,y,z))
% \end{align*}
The internal variables are expanded into series by
\begin{align}
Y_i=&B(\eta_i,hF,y,y):=\sum_{t\in T^b} \eta_i(t) \frac{h^{\rho(t)}}{\sigma(t)}G_F(t)(y,y)\\
Z_i(\lambda)=&C(\zeta^{\lambda}_i,hF,y,z):=\sum_{t\in T^w} \zeta^{\lambda}_i(t) \frac{h^{\rho(t)}}{\sigma(t)} G_F(t)(y,z)\\
\end{align}
The series coefficients $\eta_i$  map $T^b$ to $\real$, where the $\zeta_i^{\lambda}$ maps $T^b$  to polynomials over $\real$. Alternatively, we can interpret the B series defined by $\zeta^{\lambda}_i$ being dependent on a parameter.
Note, that we have $\eta_i(\emptyset)=\zeta^{\lambda}_i(\emptyset)=1$.  
 
 We consider the expansion of $hF(Y,Z)$ in a B-series when $Y, Z$ are given by B-series with coefficents $\eta, \zeta^{\lambda}$ whereas always $\eta(\emptyset)=\zeta(\emptyset)=1$.
The resulting B-series is denoted by $D(\eta,\zeta^{\lambda})$.
\begin{satz}
For a tree $t= [t_{b1}\ldots t_{bk},t_{w1}\ldots t_{wl}]_b$ we have
\begin{align}
hF(Y,Z)=&C(D(\eta, \zeta^{\lambda}),hF,y,z)
\end{align}
whereas 
\begin{align}
D(\eta, \zeta))(t)=&\prod_i \eta(t_{bi}) \prod_j \zeta^{\lambda}(t_{wj})
\end{align}
\end{satz}
The proof requires only an exact calculation of the occurence of the tree $t$ in the Taylor expansion of $hF$.

Proof:
We denote by $F^{(n)}$ the full tensor representing the $n$-th derivative of $f$ with respect to a vectorial column variable $(Y;Z)$. 
\begin{align*}
hF(Y,Z)=&hF(y,z)+h\sum_{n=1}^{\infty} \frac{1}{n!} F^{(n)}\left[\begin{pmatrix} Y-y \\ Z-z \end{pmatrix},\dots \right]\\
=&hF(y,z)+h\sum_{n=1}^{\infty} \sum_{k=0}^n   \frac{1}{n!} \begin{pmatrix} n\\ k \end{pmatrix}   F^n_{(k,n-k)}\left[Y-y, \dots;  Z-z, \dots \right]\\
%=&hF(y,z)+h\sum_{n=1}^{\infty} \sum_{k=0}^n   \frac{1}{n!} \begin{pmatrix} n\\ k \end{pmatrix}   F^n_{(k,n-k)}\left[Y-y, \dots;  Z-z, \dots \right]\\
=&hF(y,z)+h\sum_{n=1}^{\infty} \sum_{k=0}^n   \frac{1}{k! (n-k)!}   F^n_{(k,n-k)}\left[Y-y, \dots;  Z-z, \dots \right]\\
\end{align*}
Now assume we want to count the number of occurrences of a tree $t=[t_{b1},\dots, t_{bk}, t_{w1},\dots, t_{wl}]$ where $t_{b1},\dots, t_{bk} \in T^b$, $t_{w1},\dots, t_{wl}\in Tw$.
Assume that the multiplicities in which children trees occur  are $\mu_1,\dots, \mu_r$. Then, such a tree occurs in the Taylor expansion exactly
\[ \frac{k! l!}{\mu_1!\cdot \dots\cdot \mu_r!} \]
times. If all subtrees are distinct we have $k! l!$ occurrences, but for each multiple occurrence $\mu_i$ we have to deduct a factor $1/\mu_i!$.
Taken into consideration the recursion formula for the expressions $\sigma(t)$, we end up with the proposition
\begin{align}
D(\eta, \zeta^{\lambda}))(t)=&\prod_i \eta(t_{bi}) \prod_j \zeta^{\lambda}(t_{wj})
\end{align}





The algorithm below computes the series coefficients for an algorithm where we have no shift in the initial conditions.

\begin{algorithm}
\caption{Calculate series coefficient for tree $t=[t_1t_2\ldots t_m]_w$}
\begin{algorithmic}
\STATE $t=[t_1t_2\ldots t_m]_w$
\FOR{$i=1$ \TO $ns+1$} 
\STATE $\phi_i^\lambda=0$
\FOR{$j=1$ \TO $i-1$}
\STATE  $r_{ij}^\lambda=1$
\FOR{$s \in [t_1t_2\ldots t_m]$}
\IF{$s \in T^w$}
\STATE $r_{ij}^\lambda=r_{ij}^\lambda*\zeta_{i,\lambda}(s)$ Comment: Polynomial multiplication
\ELSE
\STATE $r_{ij}^\lambda=r_{ij}^\lambda*\eta_j(s)$ Comment: Scalar multiplication
\ENDIF
\ENDFOR
\FOR{$p=0$ \TO $\rho_i$}
\STATE $\phi_i^\lambda=\phi_i^\lambda+a_{ijp}\lambda^p*r_{ij}^\lambda$
\ENDFOR
\ENDFOR
\STATE Integrate $\frac{d}{d\lambda}\zeta_i^\lambda(t))=\phi_i^\lambda $
\STATE $\zeta^{\lambda 0}_i(t)=0$
\STATE $\eta_i(t^b)=\zeta^{1}_{i}(t)$ 
\ENDFOR
\end{algorithmic}
\end{algorithm}

\section{Series coefficients}
Let us compute the first coefficients of both series and the intermediate series $\phi$ for all stages. 
%We will present two version. In the first version the simplified conditions are applied as often as possible. The second version makes no further assumption with respect to the coefficients.
In addition to the standard case we also add the differentials for general additive splitting and linear additive splitting.

For a compact notification we define also
\begin{align*}
a_{ij}^{[0]}[\lambda] & = \sum_{p=0}^{\rho_i}a_{ij}^p\frac{p!}{(p+0)!}\lambda^{p+0}\\
a_{ij}^{[1]}[\lambda] & = \sum_{p=0}^{\rho_i}a_{ij}^p\frac{p!}{(p+1)!}\lambda^{p+1}\\
a_{ij}^{[m]}[\lambda] & = \sum_{p=0}^{\rho_i}a_{ij}^p\frac{p!}{(p+m)!}\lambda^{p+m}\\
\end{align*}
A further notation is the product of polynomial
\begin{align*}
(qr)[\lambda]=q[\lambda]  r[\lambda] = &(\sum_{p=0}^{\rho_1} q^p \lambda^p)(\sum_{p=0}^{\rho_2} r^p \lambda^p)\\
  = &  q^{\rho_1}p^{\rho_2}\lambda^{\rho_1+\rho_2}\\
 + & (q^{\rho_1}r^{\rho_2-1}+q^{\rho_1-1}r^{\rho_2})\lambda^{\rho_1+\rho_2-1}\\
  + & (q^{\rho_1}r^{\rho_2-2}+q^{\rho_1-1}r^{\rho_2-1}+q^{\rho_1-1}r^{\rho_2-1})\lambda^{\rho_1+\rho_2-2}\\
  + &\ldots \\
   + & (q^{0}r^{1}+q^{1}r^{0})\lambda^1\\
  + & q^{0}r^{0}\lambda^0\\
 = & \sum_{p=0}^{\rho_1+\rho_2}(\sum_{s=0}^{k} q^{l}r^{p-s})\lambda^{p}
\end{align*}
where $q^p$ is zero whenever  $p\not\in [0,\rho_1]$, resp. $r^p$ is zero whenever  $p\not\in [0,\rho_2]$.

$$
(qr)^p=\sum_{s=0}^{p}q^{s}q^{p-s}
$$

First order:

Differential 1, $F$,  $F$ , N:
\begin{align*}
r_{ij}^\lambda(\tau_w) &=1\\
\phi_i^\lambda(\tau_w) & = \SiL\\
\zeta_i^\lambda(\tau_w) & = \SiIL\\
\eta_i(\tau_b) & = \SiIO\\
\eta_j(\tau_b) & =\SjIO\\
\end{align*}

\newpage
Second order:

Differential 2, $F^{(1)}_{(1,0)}F$, $f'F$, $N'N$:
\begin{align*}
r_{ij}^\lambda([\tau_b]_w) &=\eta_j^\lambda (\tau_b)=\SjIO\\\
\phi_i^\lambda([\tau_b]_w) & = \SiL r_{ij}^\lambda ([\tau_b]_w)\\
& = \SiL \SjIO \\
\zeta_i^\lambda([\tau_b]_w) &=\SiIL \SjIO \\
\eta_i([\tau_b]_b) &=\SiIO \SjIO\\
\eta_j([\tau_b]_b) &=\SjIO \SkIO\\
\end{align*}

Differential 3, $F^{(1)}_{(0,1)}F$, $g'F$, $LN$:
\begin{align*}
r_{ij}^\lambda([\tau_w]_w) &=\zeta_i(\tau_w)=\SiIL \\
\phi_i^\lambda([\tau_w]_w) &= \SiL r_{ij}^\lambda ([\tau_w]_w)\\
& = \SiL \SiIL\\\
\zeta_i^\lambda([\tau_w]_w) &= (\Si \SjI )\EIL \\
\eta_i([\tau_w]_b) &=(\Si \SjI )\EIO \\
\eta_j([\tau_w]_b) &=(\Sj \SkI)\EIO \\
\end{align*}

\newpage
Third order:

Differential 4, $F^{(2)}_{(2,0)}FF$, $f''FF$, $N''NN$ :
\begin{align*}
r_{ij}^\lambda([\tau_b,\tau_b]_w) & = (\eta_j(\tau_b))^2=( \SjIO)^2\\
\phi_i^\lambda([\tau_b,\tau_b]_w) &= \SiL r_{ij}^\lambda ([\tau_b,\tau_b]_w)\\
& = \SiL( \SjIO)^2\\
\zeta_i^\lambda([\tau_b,\tau_b]_w) &= \SiIL ( \SjIO)^2\\
\eta_i([\tau_b,\tau_b]_b)) &=\SiIO ( \SjIO)^2\\
\eta_j([\tau_b,\tau_b]_b)) &=\SjIO ( \SkIO)^2\\
\end{align*}

Differential 5, $F^{(2)}_{(1,1)}FF$:
\begin{align*}
r_{ij}^\lambda([\tau_w,\tau_b]_w & =\zeta_i^\lambda(\tau_w) \eta_j(\tau_b)= ( \SiIL)(\SjIO)\\
\phi_i^\lambda([\tau_w,\tau_b]_w) &= \SiL r_{ij}^\lambda ([\tau_w,\tau_b]_w)\\
& = (\SiIL) (\SiL\SjIO) \\
& = ((\SiI) (\Si\SjIO))\L\\
\zeta_i^\lambda([\tau_w,\tau_b]_w) & =((\SiI) (\Si\SjIO))\EIL\\
\eta_i([\tau_w,\tau_b]_b)) & = ((\SiI) (\Si\SjIO))\EIO\\
\eta_j([\tau_w,\tau_b]_b)) & = ((\SjI) (\Sj\SkIO))\EIO\\
\end{align*}

Differential 6, $F^{(2)}_{(0,2)}FF$, $g''FF$:
\begin{align*}
r_{ij}^\lambda([\tau_w,\tau_w]_w)& =(\zeta_i^\lambda(\tau_w))^2= (\SiIL)^2\\
\phi_i^\lambda([\tau_w,\tau_w]_w) &= \SiL ([\tau_w,\tau_w]_w)\\
& = \SiL(\SiIL)^2\\
\zeta_i^\lambda([\tau_w,\tau_w]_w) &=(\Si(\SiI)^2)\EIL\\
\eta_i([\tau_w,\tau_w]_b)) &=(\Si(\SiI)^2)\EIO\\
\eta_j([\tau_w,\tau_w]_b)) &=(\Sj(\SjI)^2)\EIO
\end{align*}


%[[\tau_b]_b]_b
Differential 7, $F^{(1)}_{(1,0)}F^{(1)}_{(1,0)}F$, $f'f'F$ , $N'N'N$:
\begin{align*}
r_{ij}^\lambda([[\tau_b]_b]_w)& =\eta_j([\tau_b]_b)= \SjIO \SkIO \\
\phi_i^\lambda([[\tau_b]_b]_w) &= \SiL r_{ij}^\lambda ([[\tau_b]_b]_w)\\
& = \SiL \SjIO \SkIO \\
\zeta_i^\lambda([[\tau_b]_b]_w) &=\SiIL \SjIO \SkIO \\
\eta_i([[\tau_b]_b]_b)) &=\SiIO \SjIO \SkIO
\end{align*}

Differential 8, $F^{(1)}_{(1,0)}F^{(1)}_{(0,1)}F$, $f'g'F$, $N'LN$:
\begin{align*}
r_{ij}^\lambda([[\tau_w]_b]_w)& =\eta_j([\tau_w]_b)= (\Sj \SkI)\EIO \\
\phi_i^\lambda([[\tau_w]_b]_w) &= \SiL r_{ij}^\lambda ([[\tau_w]_b]_w)\\
& = \SiL (\Si \SjI)\EIO\\
\zeta_i^\lambda([[\tau_w]_b]_w)& = \SiIL (\S \SkI)\EIO\\
\eta_i([[\tau_w]_b]_b)& = \SiIO (\Sj \SkI)\EIO\\
\eta_j([[\tau_w]_b]_b)& =\SjIO (\Sk \SlI)\EIO
\end{align*}

%[[\tau_b]_w]_b
Differential 9, $LN'N$:
\begin{align*}
r_{ij}^\lambda([[\tau_b]_w]_w)& =\zeta_i^\lambda([[\tau_b]_w])=\SiIL \SjIO\\
\phi_i^\lambda([[\tau_b]_w]_w)) &= \SiL r_{ij}^\lambda ([[\tau_b]_w]_w))\\
&=\SiL \SiIL \SjIO\\
\zeta_i^\lambda([[\tau_b]_w]_w)& =(\Si \SiI \SjIO)\EIL\\
\eta_i([[\tau_b]_w]_b)& = (\Si \SiI \SjIO)\EIO\\
\eta_j([[\tau_b]_w]_b)& =(\Sj \SjI\SkIO)\EIO\\
\end{align*}

Differential 10, $F^{(1)}_{(0,1)}F^{(1)}_{(0,1)}F$, $g'g'F$, $LLN$:
%[[\tau_w]_w]_b
\begin{align*}
r_{ij}^\lambda([[\tau_w]_w]_w)& =\zeta_i^\lambda([\tau_w]_w)=(\Si \SjI )\EIL\\
\phi_i^\lambda([[\tau_w]_w]_w) &= \SiL r_{ij}^\lambda ([[\tau_w]_w]_w))\\
&= \SiL (\Si \SjI )\EIL\\
\zeta_i^\lambda([[\tau_w]_w]_w)& =(\Si (\Si \SjI )\EI)\EIL \\
\eta_i([[\tau_w]_w]_b)& = (\Si (\Si \SjI )\EI)\EIO\\
\eta_j([[\tau_w]_w]_b)& = (\Sj (\Sj \SkI )\EI)\EIO
\end{align*}

Fourth order:

Differential 11, $F^{(3)}_{(3,0)}FFF$, $f'''FFF$, $N'''NNN$:
 % [\tau_b,\tau_b,\tau_b]_b$ 
\begin{align*}
r_{ij}^\lambda([\tau_b,\tau_b,\tau_b]_w)& =(\eta_j(\tau_b))^3=(\SiIO)^3\\
\phi_i^\lambda([\tau_b,\tau_b,\tau_b]_w) &= \SiL r_{ij}^\lambda ([\tau_b,\tau_b,\tau_b]_w))\\
&= \SiL (\SiIO)^3\\
\zeta_i^\lambda([\tau_b,\tau_b,\tau_b]_w)& = \SiIL (\SiIO)^3 \\
\eta_i([\tau_b,\tau_b,\tau_b]_b)& = (\SiIO)^4\\
\eta_j([\tau_b,\tau_b,\tau_b]_b)& = (\SjIO)^4
\end{align*}

Differential 12, $F^{(3)}_{(2,1)}FFF$:
%[\tau_b,\tau_b,\tau_w]_b
%\zeta_i^\lambda(\tau_w) & = \SiIL
\begin{align*}
r_{ij}^\lambda([\tau_b,\tau_b,\tau_w]_w)& =\zeta_i^\lambda(\tau_w)(\eta_j(\tau_b))^2=\SiIL(\SiIO)^2\\
\phi_i^\lambda([\tau_b,\tau_b,\tau_w]_w) &= \SiL r_{ij}^\lambda ([\tau_b,\tau_b,\tau_w]_w))\\
&= \SiL\SiIL(\SiIO)^2\\
\zeta_i^\lambda([\tau_b,\tau_b,\tau_w]_w)& = (\Si\SiI)\EIL(\SiIO)^2 \\
\eta_i([\tau_b,\tau_b,\tau_w]_b)& = (\Si\SiI)\EIO(\SiIO)^2\\
\eta_j([\tau_b,\tau_b,\tau_w]_b)& = (\Sj\SjI)\EIO(\SjIO)^2
\end{align*}

Differential 13, $F^{(3)}_{(1,2)}FFF$:
%[\tau_b,\tau_w,\tau_w]_b
\begin{align*}
r_{ij}^\lambda([\tau_b,\tau_w,\tau_w]_w)& =(\zeta_i^\lambda(\tau_w))^2(\eta_j\tau_b)=(\SiIL)^2\SjIO)\\
\phi_i^\lambda([\tau_b,\tau_w,\tau_w]_w) &= \SiL r_{ij}^\lambda ([\tau_b,\tau_w,\tau_w]_w))\\
&= \SiL(\SiIL)^2 \SjIO\\
\zeta_i^\lambda([\tau_b,\tau_w,\tau_w]_w) & = (\Si( \SiI)^2) \SjIO)\EIL\\
\eta_i([\tau_b,\tau_w,\tau_w]_b)& = (\Si( \SiI)^2) \SjIO)\EIO\\
\eta_j([\tau_b,\tau_w,\tau_w]_b)& = (\Sj( \SjI)^2) \SkIO)\EIO
\end{align*}

Differential  14, $F^{(3)}_{(0,3)}FF$, $g'''FFF$:
%[\tau_w,\tau_w,\tau_w]_b 
\begin{align*}
r_{ij}^\lambda([\tau_w,\tau_w,\tau_w]_w)& =(\zeta_i^\lambda(\tau_w))^3=(\SiIL)^3\\
\phi_i^\lambda([\tau_w,\tau_w,\tau_w]_w) &= \SiL r_{ij}^\lambda ([\tau_w,\tau_w,\tau_w]_w))\\
&= \SiL(\SiIL)^3 \\
\zeta_i^\lambda([\tau_w,\tau_w,\tau_w]_w) & = (\Si( \SiI)^3)\EIL \\
\eta_i([\tau_w,\tau_w,\tau_w]_b)& = (\Si (\SiI)^3)\EIO \\
\eta_j([\tau_w,\tau_w,\tau_w]_b)& = (\Sj (\SjI)^3)\EIO
\end{align*}

Differential  15, $F^{(2)}_{(2,0)}F^{(1)}_{(1,0)}FF$, $f''f'FF$, $N''N'NN$ 
%[\tau_b,[\tau_b]_b]_b
\begin{align*}
r_{ij}^\lambda([\tau_b,[\tau_b]_b]_w)& =\eta_j(\tau_b)\eta_j([\tau_b]_b)=\SjIO(\SjIO \SkIO)\\
\phi_i^\lambda([\tau_b,[\tau_b]_b]_w) &= \SiL r_{ij}^\lambda ([\tau_w,\tau_w,\tau_w]_w))\\
&= \SiL\SjIO(\SjIO \SkIO) \\
\zeta_i^\lambda([\tau_b,[\tau_b]_b]_w) & =  \SiIL\SjIO (\SjIO \SkIO) \\
\eta_i([\tau_b,[\tau_b]_b]_b)& = \SiIO\SjIO(\SjIO \SkIO)\\
\eta_j([\tau_b,[\tau_b]_b]_b)& = \SjIO\SkIO(\SkIO \SlIO)
\end{align*}

Differential  16, $F^{(2)}_{(1,1)}F^{(1)}_{(1,0)}FF$
%[\tau_b,[\tau_b]_w]_b
\begin{align*}
r_{ij}^\lambda([\tau_b,[\tau_b]_w]_w)& =\eta_j(\tau_b)\zeta_i^\lambda([\tau_b]_w)=\SjIO(\SiIL \SjIO)\\
\phi_i^\lambda([\tau_b,[\tau_b]_w]_w) &= \SiL r_{ij}^\lambda ([\tau_b,[\tau_b]_w]_w)\\
&= (\SiL\SjIO)(\SiIL \SjIO) \\
\zeta_i^\lambda([\tau_b,[\tau_b]_w]_w) & = ((\Si\SjIO)(\SiI \SjIO))\EIL  \\
\eta_i([\tau_b,[\tau_b]_w]_b)& =  ((\Si\SjIO)(\SiI \SjIO))\EIO\\
\eta_j([\tau_b,[\tau_b]_w]_b)& = ((\Sj\SkIO)(\SjI \SkIO))\EIO
\end{align*}

Differential  17, $F^{(2)}_{(2,0)}F^{(1)}_{(0,1)}FF$
%[\tau_w,[\tau_b]_b]_b
\begin{align*}
r_{ij}^\lambda([\tau_w,[\tau_b]_b]_w)& =\zeta_i^\lambda(\tau_w)\eta_j([\tau_w]_w)=\SiIL\SjIO \SkIO\\
\phi_i^\lambda([\tau_w,[\tau_b]_b]_w) &= \sum_{j=1}^{i-1}a_{ij}^{[0]}[\lambda]r_{ij}^\lambda ([\tau_w,[\tau_b]_b]_w)\\
\end{align*}

Differential 18, $F^{(2)}_{(2,0)}F^{(1)}_{(0,1)}FF$, $f''g'FF $, $N''LNN$ 
%$[\tau_b,[\tau_w]_b]_b$ \\
\begin{align*}
r_{ij}^\lambda([\tau_b,[\tau_w]_b]_w)& =\eta_j^\lambda(\tau_b)\eta_j^\lambda([\tau_w]_b)=\sum_{k=1}^{j-1}a_{jk}^{[1]}[1](\sum_{k=1}^{j-1}a_{jk}^{[0]}\sum_{k=1}^{j-1}a_{jk}^{[1]})^{[1]}[1]\\
\phi_i^\lambda([\tau_b,[\tau_w]_b]_w) &= \sum_{j=1}^{i-1}a_{ij}^{[0]}[\lambda]r_{ij}^\lambda ([[\tau_w]_w]_w)\\
&= \sum_{j=1}^{i-1}a_{ij}^{[0]}[\lambda]\sum_{k=1}^{j-1}a_{jk}^{[1]}[1](\sum_{k=1}^{j-1}a_{jk}^{[0]}\sum_{k=1}^{j-1}a_{jk}^{[1]})^{[1]}[1]\\
\zeta_i^\lambda([\tau_b,[\tau_w]_b]_w) & = (\sum_{j=1}^{i-1}a_{ij}^{[0]}\sum_{k=1}^{j-1}a_{jk}^{[1]}[1](\sum_{k=1}^{j-1}a_{jk}^{[0]}\sum_{k=1}^{j-1}a_{jk}^{[1]})^{[1]}[1])^{[1]}[\lambda]\\
\eta_i([\tau_b,[\tau_w]_b]_w) & = (\sum_{j=1}^{i-1}a_{ij}^{[0]}\sum_{k=1}^{j-1}a_{jk}^{[1]}[1](\sum_{k=1}^{j-1}a_{jk}^{[0]}\sum_{k=1}^{j-1}a_{jk}^{[1]})^{[1]}[1])^{[1]}[1]\\
\end{align*}

Differential 24, $F^{(1)}_{(0,1)}F^{(2)}_{(2,0)}FF$, $g'f''FF$,  $LN''NN$:
%& $[[\tau_b,\tau_b]_w]_b$
\begin{align*}
r_{ij}^\lambda([[\tau_b,\tau_b]_w]_w)& =\zeta_i^{\lambda}([\tau_b,\tau_b]_w)) =\sum_{j=1}^{i-1}a_{ij}^{[1]}[\lambda](\sum_{k=1}^{j-1} a_{jk}^{[1]}[1])^2\\
\phi_i^\lambda([[\tau_b,\tau_b]_w]_w) &= \sum_{j=1}^{i-1}a_{ij}^{[0]}[\lambda]r_{ij}^\lambda ([\tau_b,\tau_b]_b)_w)\\
& = \sum_{j=1}^{i-1}a_{ij}^{[0]}[\lambda]\sum_{j=1}^{i-1}a_{ij}^{[1]}[\lambda](\sum_{k=1}^{j-1} a_{jk}^{[1]}[1])^2\\
\zeta_i^\lambda([[\tau_b,\tau_b]_w]_w) & =(\sum_{j=1}^{i-1}a_{ij}^{[0]}\sum_{j=1}^{i-1}a_{ij}^{[1]})^{[1]}[\lambda](\sum_{k=1}^{j-1} a_{jk}^{[1]}[1])^2\\
\eta_i([[\tau_b,\tau_b]_w]_b) & =(\sum_{j=1}^{i-1}a_{ij}^{[0]}\sum_{j=1}^{i-1}a_{ij}^{[1]})^{[1]}[1](\sum_{k=1}^{j-1} a_{jk}^{[1]}[1])^2\\
\end{align*}

Differential  30, $F^{(1)}_{(0,1)}F^{(1)}_{(1,0)}F^{(1)}_{(1,0)}F$, $g'f'f'F$, $LN'N'N$ 
%[[[\tau_w]_b]_b]_b$ Falsch
\begin{align*}
r_{ij}^\lambda([[[\tau_b]_b]_w]_w)& =\zeta_i^{\lambda}([[\tau_b]_b]_w])=\sum_{j=1}^{i-1}a_{ij}^{[1]}[\lambda]\sum_{k=1}^{j-1}a_{jk}^{[1]}[1]\sum_{l=1}^{k-1}a_{kl}^{[1]}[1]\\
\phi_i^\lambda([[[\tau_w]_b]_b]_w)&= \sum_{j=1}^{i-1}a_{ij}^{[0]}[\lambda]r_{ij}^\lambda ([[[\tau_b]_b]_w]_w)\\
& = \sum_{j=1}^{i-1}a_{ij}^{[0]}[\lambda]\sum_{j=1}^{i-1}a_{ij}^{[1]}[\lambda]\sum_{k=1}^{j-1}a_{jk}^{[1]}[1]\sum_{l=1}^{k-1}a_{kl}^{[1]}[1]\\
\zeta_i^\lambda([[[\tau_b]_b]_w]_w)& =(\sum_{j=1}^{i-1}a_{ij}^{[0]}\sum_{j=1}^{i-1}a_{ij}^{[1]})^{[1]})[\lambda]\sum_{k=1}^{j-1}a_{jk}^{[1]}[1]\sum_{l=1}^{k-1}a_{kl}^{[1]}[1]\\
\eta_i([[[\tau_b]_b]_w]_b)& =(\sum_{j=1}^{i-1}a_{ij}^{[0]}\sum_{j=1}^{i-1}a_{ij}^{[1]})^{[1]})[1]\sum_{k=1}^{j-1}a_{jk}^{[1]}[1]\sum_{l=1}^{k-1}a_{kl}^{[1]}[1]\
\end{align*}

Differential 31, $F^{(1)}_{(1,0)}F^{(1)}_{(0,1)}F^{(1)}_{(1,0)}F$, $f'g'f'F$,  $N'LN'N$
 %$[[[\tau_b]_w]_b]_w$ &
 \begin{align*}
r_{ij}^\lambda([[[\tau_b]_w]_b]_w)& =\eta_j([[\tau_b]_w]_b) =( \sum_{k=1}^{j-1}a_{ij}^{[0]}\sum_{k=1}^{j-1}a_{jk}^{[1]}\sum_{l=1}^{k-1} a_{kl}^{[1]}[1])^{[1]}[1]\\
\phi_i^\lambda([[[\tau_b]_w]_b]_w)&= \sum_{j=1}^{i-1}a_{ij}^{[0]}[\lambda]r_{ij}^\lambda ([[[\tau_b]_w]_b]_w)\\
& = \sum_{j=1}^{i-1}a_{ij}^{[0]}[\lambda]( \sum_{k=1}^{j-1}a_{ij}^{[0]}\sum_{k=1}^{j-1}a_{jk}^{[1]}\sum_{l=1}^{k-1} a_{kl}^{[1]}[1])^{[1]}[1]\\
\zeta_i^\lambda([[[\tau_b]_w]_b]_w) & =\sum_{j=1}^{i-1}a_{ij}^{[1]}[\lambda]( \sum_{k=1}^{j-1}a_{ij}^{[0]}\sum_{k=1}^{j-1}a_{jk}^{[1]}\sum_{l=1}^{k-1} a_{kl}^{[1]}[1])^{[1]}[1]\\
\eta_i([[[\tau_b]_w]_b]_b) & =\sum_{j=1}^{i-1}a_{ij}^{[1]}[1]( \sum_{k=1}^{j-1}a_{ij}^{[0]}\sum_{k=1}^{j-1}a_{jk}^{[1]}\sum_{l=1}^{k-1} a_{kl}^{[1]}[1])^{[1]}[1]\\
\end{align*}

Differential 32, $F^{(1)}_{(1,0)}F^{(1)}_{(1,0)}F^{(1)}_{(0,1)}F$, $f'f'g'F$, $N'N'LN$:
%[[[\tau_b]_b]_w]_b
\begin{align*}
r_{ij}^\lambda([[[\tau_b]_b]_w]_w)& =\zeta_i^{\lambda}([[\tau_b]_b]_w])=\sum_{j=1}^{i-1}a_{ij}^{[1]}[\lambda]\sum_{k=1}^{j-1}a_{jk}^{[1]}[1]\sum_{l=1}^{k-1}a_{kl}^{[1]}[1]\\
\phi_i^\lambda([[[\tau_w]_b]_b]_w)&= \sum_{j=1}^{i-1}a_{ij}^{[0]}[\lambda]r_{ij}^\lambda ([[[\tau_b]_b]_w]_w)\\
& = \sum_{j=1}^{i-1}a_{ij}^{[0]}[\lambda]\sum_{j=1}^{i-1}a_{ij}^{[1]}[\lambda]\sum_{k=1}^{j-1}a_{jk}^{[1]}[1]\sum_{l=1}^{k-1}a_{kl}^{[1]}[1]\\
\zeta_i^\lambda([[[\tau_b]_b]_w]_w)& =(\sum_{j=1}^{i-1}a_{ij}^{[0]}\sum_{j=1}^{i-1}a_{ij}^{[1]})^{[1]})[\lambda]\sum_{k=1}^{j-1}a_{jk}^{[1]}[1]\sum_{l=1}^{k-1}a_{kl}^{[1]}[1]\\
\eta_i([[[\tau_b]_b]_w]_b)& =(\sum_{j=1}^{i-1}a_{ij}^{[0]}\sum_{j=1}^{i-1}a_{ij}^{[1]})^{[1]})[1]\sum_{k=1}^{j-1}a_{jk}^{[1]}[1]\sum_{l=1}^{k-1}a_{kl}^{[1]}[1]\
\end{align*}


\section{Adding shifted initial conditions}
Algorithm:

\begin{algorithm}
\caption{Calculate series coefficient for tree $t$}
\begin{algorithmic}
\STATE $t=[t_1t_2\ldots t_m]$
\FOR{$i=1$ \TO $ns+1$} 
\STATE $\phi_i^\lambda=0$
\FOR{$j=1$ \TO $i-1$}
\STATE  $r_{ij}^\lambda=1$
\FOR{$s \in [t_1t_2\ldots t_m]$}
\IF{$s \in T^w$}
\STATE $r_{ij}^\lambda=r_{ij}^\lambda*\zeta_{i,\lambda}(s)$ Comment: Polynomial multiplication
\ELSE
\STATE $r_{ij}^\lambda=r_{ij}^\lambda*\eta_j(s)$ Comment: Scalar multiplication
\ENDIF
\ENDFOR
\FOR{$p=0$ \TO $\rho_i$}
\STATE $\phi_i^\lambda=\phi_i^\lambda+a_{ijp}\lambda^p*r_{ij}^\lambda$
\ENDFOR
\ENDFOR
\STATE Integrate $\frac{d}{d\lambda}\zeta_i^\lambda(t))=\phi_i^\lambda $
\STATE $\zeta_i^0(t)=0$
\FOR{$j=0$ \TO $i-1$}
\STATE $\zeta_i^0(t)=\zeta_i^0(t)+d_{ij}\eta_j(t)$
\ENDFOR
\STATE $\eta_i(t)=\zeta_{i}^1(t)$ 
\ENDFOR
\end{algorithmic}
\end{algorithm}

An identity
$$
DR=D(I-D)^{-1}=(D-I+I)(I-D)^{-1}=-I+(I-D)^{-1}=-I+R
$$

First order:

Differential 1, $F$,  $F$ , N:
\begin{align*}
r_{ij}^\lambda(\tau_w) &=1\\
\phi_i^\lambda(\tau_w) & = \SiL\\
\zeta_i^\lambda(\tau_w) & = \SiIL+\Di\eta_j(\tau_b)\\
\eta_i(\tau_b) & = \SiIO+\Di\eta_j(\tau_b)\\
\eta_i(\tau_b) & = \Ri \SjIO\\
\eta_j(\tau_b) & = \Rj\SkIO\\
\zeta_i^\lambda(\tau_w) & = \SiIL+\Di\Rj\SkIO\
\end{align*}

Second order:

Differential 2, $F^{(1)}_{(1,0)}F$, $f'F$, $N'N$:
\begin{align*}
r_{ij}^\lambda([\tau_b]_w) &=\eta_j^\lambda (\tau_b)=\Rj\SkIO\\\
\phi_i^\lambda([\tau_b]_w) & = \SiL r_{ij}^\lambda ([\tau_b]_w)\\
& = \SiL \Rj\SkIO \\
\zeta_i^\lambda([\tau_b]_w) &=\SiIL \Rj\SkIO +\sum_{j=1}^{i-1}d_{ij}\eta_j([\tau_b]_b)\\
\eta_i([\tau_b]_b) &=\Ri  \SjIO \Rk\SlIO\\
\eta_j([\tau_b]_b) &=\Rj  \SkIO \Rl\SrIO\\
\end{align*}

Differential 3, $F^{(1)}_{(0,1)}F$, $g'F$, $LN$:
\begin{align*}
r_{ij}^\lambda([\tau_w]_w) &=\zeta_i(\tau_w)=\SiIL+\Di\Rj\SkIO\\
\phi_i^\lambda([\tau_w]_w) &= \SiL r_{ij}^\lambda ([\tau_w]_w)\\
& = \SiL (\SiIL+\Di \Rj\SkIO)\\\
\zeta_i^\lambda([\tau_w]_w) &=\frac{1}{2} (\SiIL )^2+\SiIL \Di \Rj\SkIO+\Di \eta_j([\tau_w]_b))\\
\eta_i([\tau_w]_b) &= \frac{1}{2} (\SiIO )^2+\SiIO \Di \Rj\SkIO+\Di \eta_j([\tau_w]_b))\\
\eta_i([\tau_w]_b) &=\Ri( \frac{1}{2} (\SjIO )^2+\SjIO \Dj \Rk\SlIO) \\
& = \Ri( \frac{1}{2} (\SjIO )^2-\SjIO\SjIO-\SjIO\Rj\SkIO)\\
\eta_j([\tau_w]_b) &=(\Sj \SkI)\EIO \\
\end{align*}




\vspace{.5cm} 
\begin{landscape}
\begin{tabular}{lccccccccc}
\hline
 & El. Diff. & El. Diff. & El. Diff. &t & Notation & Notation & $\sigma(t)$ & Order Conditions & Order Owren\\
\hline
1 & $F$ &  $F$ & N &\treea & $\tau_b$ & $.$ &1 & $\SSiIO$ &\\
\hline
2 & $F^{(1)}_{(1,0)}F$ &$f'F$ & $N'N$ &\treeb & $[\tau_b]_b$ & $[.]$ &1 &$\SSiIO \SSjIO $&\\
3 & $F^{(1)}_{(0,1)}F$ & $g'F$ & $LN$ &\treew & $[\tau_w]_b$ & $[o]$ &1 &$(\SSi \SSjI )\EIO$ &\\
\hline
4& $F^{(2)}_{(2,0)}FF$ & $f''FF$ & $N''NN$ &\treec & $[\tau_b,\tau_b]_b$ & $[.,.]$ &1 &$\SSiIO ( \SSjIO)^2$ &\\
5 & $F^{(2)}_{(1,1)}FF$ & $$ & $$ &\treecc & $[\tau_w,\tau_b]_b$ & $[o,.]$ & 2 &$((\SSiI) (\SSi\SSjIO))\EIO$ &\\
6 & $F^{(2)}_{(0,2)}FF$ & $g''FF$ & $$ &\treeccc & $[\tau_w,\tau_w]_b$ & $[o,o]$ & 1& $(\SSi(\SSiI)^2)\EIO$& \\
7 & $F^{(1)}_{(1,0)}F^{(1)}_{(1,0)}F$ & $f'f'F$ & $N'N'N$ &\treed & $[[\tau_b]_b]_b$ &$[[.]]$ & 1 & $\SSiIO \SSjIO \SSkIO$  & \\
8 & $F^{(1)}_{(1,0)}F^{(1)}_{(0,1)}F$ & $f'g'F$ & $N'LN$ &\treedd & $[[\tau_w]_b]_b$ & $[[o]]$ &1&$\SSiIO (\SSj \SSkI)\EIO$ & \\
9 & $F^{(1)}_{(0,1)}F^{(1)}_{(1,0)}F$ & $g'f'F$ & $LN'N$ &\treeddd & $[[\tau_b]_w]_b$ & $[(.)]$ &1 & $(\SSi \SSiI \SSjIO)\EIO$ & \\
10 & $F^{(1)}_{(0,1)}F^{(1)}_{(0,1)}F$ & $g'g'F$ & $LLN$ &\treedddd & $[[\tau_w]_w]_b$ & $[(\o)]$ & 1 & $(\SSi (\SSi \SSjI )\EI)\EIO$ &\\
\hline
\end{tabular}
\end{landscape}

\begin{landscape}
\begin{tabular}{lccccccccc}
\hline
 & El. Diff. & El. Diff. & El. Diff. &t & Notation & Notation & $\sigma(t)$ & Order Conditions & Order Owren\\
\hline
11& $F^{(3)}_{(3,0)}FFF$& $f'''FFF$  & $N'''NNN$ & \treef & $[\tau_b,\tau_b,\tau_b]_b$ & $[.,.,.]$ & 1 & $(\SSiIO)^4$& \\
12& $F^{(3)}_{(2,1)}FFF$ & $$ &$$ &\treeff & $[\tau_b,\tau_b,\tau_w]_b$ & $[.,.,o]$ &1 & $(\SSi\SSiI)\EIO(\SSiIO)^2$ &\\
13& $F^{(3)}_{(1,2)}FFF$& $$  &$$ &\treefff & $[\tau_b,\tau_w,\tau_w]_b$ & $[.,o,o]$ & 1& $(\SSi (\SSiI)^2)\EIO \SSiIO$&\\
14& $F^{(3)}_{(0,3)}FFF$ & $g'''FFF$ &  &\treeffff & $[\tau_w,\tau_w,\tau_w]_b$ & $[o,o,o]$ &1& $(\SSi (\SSiI)^3)\EIO$ & \\
\hline
\end{tabular}
\end{landscape}

\begin{landscape}
\begin{tabular}{lccccccccc}
\hline
 & El. Diff. & El. Diff. & El. Diff. &t & Notation & Notation & $\sigma(t)$ & Order Conditions & Order Owren\\
\hline
15& $F^{(2)}_{(2,0)}F^{(1)}_{(1,0)}FF$& $f''f'FF$  &$N''N'NN$ &\treeg & $[\tau_b,[\tau_b]_b]_b$ & $[.,[.]]$ &1 &$\SSiIO\SSjIO(\SSjIO \SSkIO)$ &\\
16& $F^{(2)}_{(1,1)}F^{(1)}_{(1,0)}FF$ & $$  & $$ &\treegg & $[\tau_b,[\tau_b]_w]_b$ & $[.,(.)]$ &1 &$((\SSi\SSjIO)(\SSiI \SSjIO))\EIO$ &\\
17& $F^{(2)}_{(1,1)}F^{(1)}_{(1,0)}FF$ & $$  & $$ &\treegggg& $[\tau_w,[\tau_b]_b]_b$ & $[o,[.]]$ &1 & & \\
18& $F^{(2)}_{(2,0)}F^{(1)}_{(0,1)}FF$ & $f''g'FF $  &$N''LNN$ &\treeggg& $[\tau_b,[\tau_w]_b]_b$ & $[.,[o]]$ &1 & & \\
19& $F^{(2)}_{(1,1)}F^{(1)}_{(0,1)}FF$ & $$  &$$ &\treeggG & $[\tau_b,[\tau_w]_w]_b$ & $[.,(o)]$ &1 & &\\
20& $F^{(2)}_{(1,1)}F^{(1)}_{(1,0)}FF$ & $$  &$$ &\treeggGg& $[\tau_w,[\tau_b]_b]_b$ & $[o,[o]]$ &1 & &\\
21& $F^{(2)}_{(0,2)}F^{(1)}_{(1,0)}FF$ & $g''f'FF $  &$$ &\treegGgg& $[\tau_w,[\tau_b]_w]_b$ & $[o,(.)]$ &1 & & \\
22& $F^{(2)}_{(0,2)}F^{(1)}_{(0,1)}FF$ & $g''g'FF $  &$$ &\treegGgg& $[\tau_w,[\tau_w]_w]_b$ & $[o,(o)]$ &1 & & \\
\hline
\end{tabular}
\end{landscape}
\vspace{.5cm} 
\begin{tabular}{lccccccc}
\hline
 & El. Diff. & El. Diff. & El. Diff. &t & Notation & Notation & $\sigma(t)$\\
\hline
23& $F^{(1)}_{(1,0)}F^{(2)}_{(2,0)}FF$& $f'f''FF$  &$N'N''NN$ &\treeh & $[[\tau_b,\tau_b]_b]_b$ & $[[.,.]]$ &1\\
24& $F^{(1)}_{(0,1)}F^{(2)}_{(2,0)}FF$& $g'f''FF$  &$LN''NN$ &\treehh & $[[\tau_b,\tau_b]_w]_b$ & $[(.,.)]$ &1\\
25& $F^{(1)}_{(1,0)}F^{(2)}_{(1,1)}FF$& $$  &$$ &\treehhh& $[[\tau_b,\tau_b]_b]_b$ & $[[.,.]]$ &1\\
26& $F^{(1)}_{(1,0)}F^{(2)}_{(2,0)}FF$& $$  &$$ & \treehhhh & $[[\tau_b,\tau_b]_b]_b$ & $[[.,.]]$ &1\\
27& $F^{(1)}_{(1,0)}F^{(2)}_{(0,2)}FF$& $f'g''FF$  &$$ &\treehhhhhh & $[[\tau_w,\tau_w]_b]_b$ & $[[o,o]]$ &1\\
28& $F^{(1)}_{(1,0)}F^{(2)}_{(2,0)}FF$& $g'g''FF$  &$$ &\treehhhhh & $[[\tau_w,\tau_w]_w]_b$ & $[(o,o)]$ &1\\
\hline
\end{tabular}

\vspace{.5cm} 
\begin{tabular}{lccccccc}
\hline
 & El. Diff. & El. Diff. & El. Diff. & t & Notation & Notation & $\sigma(t)$\\
\hline
29& $F^{(1)}_{(1,0)}F^{(1)}_{(1,0)}F^{(1)}_{(1,0)}F$& $f'f'f'F$  &$N'N'N'N$ &\treei & $[[[\tau_b]_b]_b]_b$ & $[[[.]]]$ &1\\
30& $F^{(1)}_{(0,1)}F^{(1)}_{(1,0)}F^{(1)}_{(1,0)}F$& $g'f'f'F$  &$LN'N'N$ &\treeii & $[[[\tau_w]_b]_b]_b$ & $[([.])]$ &1\\
31& $F^{(1)}_{(1,0)}F^{(1)}_{(0,1)}F^{(1)}_{(1,0)}F$& $f'g'f'F$  &$N'LN'N$ &\treeiii& $[[[\tau_b]_w]_b]_b$ & $[[(.)]]$ &1\\
32& $F^{(1)}_{(1,0)}F^{(1)}_{(1,0)}F^{(1)}_{(0,1)}F$& $f'f'g'F$  &$N'N'LN$ &\treeiiii & $[[[\tau_b]_b]_w]_b$ & $[[[o]]]$ &1\\
\hline
\end{tabular}


\vspace{.5cm} 
\begin{tabular}{lccccccc}
\hline
 & El. Diff. & El. Diff. & El. Diff. &t & Notation & Notation & $\sigma(t)$\\
\hline
33& $F^{(1)}_{(0,1)}F^{(1)}_{(0,1)}F^{(1)}_{(0,1)}F$& $g'g'g'F$  &$LLLN$ &\treeI & $[[[\tau_w]_w]_w]_b$ & $[((o))]$ &1\\
34& $F^{(1)}_{(0,1)}F^{(1)}_{(0,1)}F^{(1)}_{(1,0)}F$ & $f'g'g'F$  & $N'LLN$ &\treeII & $[[[\tau_w]_w]_b]_b$ & $[[(o)]]$ &1\\
35& $F^{(1)}_{(1,0)}F^{(1)}_{(0,1)}F^{(1)}_{(1,0)}F$& $g'f'g'F$  &$LN'LN$ &\treeIII& $[[[\tau_w]_b]_w]_b$ & $[([o])]$ &1\\
36& $F^{(1)}_{0,1)}F^{(1)}_{(0,1)}F^{(1)}_{(0,1)}F$& $g'g'f'F$  &$LLN'N$ &\treeIIII& $[[[\tau_b]_w]_w]_b$ & $[((.))]$ &1\\
\hline
\end{tabular}

\vspace{2cm}






 




%A recursion
%\begin{align*}
%pq = &(\sum_{k=0}^{\rho_1} p^k \lambda^k)(\sum_{k=0}^{\rho_2} q^k \lambda^k)\\
%  = &  p^{\rho_1}q^{\rho_2}\lambda^{\rho_1+\rho_2}\\
% + & (p^{\rho_1}q^{\rho_2-1}+p^{\rho_1-1}q^{\rho_2})\lambda^{\rho_1+\rho_2-1}\\
%  + & (p^{\rho_1}q^{\rho_2-2}+p^{\rho_1-1}q^{\rho_2-1}+p^{\rho_1-1}q^{\rho_2-1})\lambda^{\rho_1+\rho_2-2}\\
%  + &\ldots \\
%   + & (p^{0}q^{1}+p^{1}q^{0})\lambda^1\\
%  + & p^{0}q^{0}\lambda^0\\
% = & \sum_{k=0}^{\rho_1+\rho_2}(\sum_{l=0}^{k} p^{l}q^{k-l})\lambda^{k}
%\end{align*}
%where $p^l$ is zero whenever  $l\not\in [0,\rho_1]$, resp. $q^l$ is zero whenever  $l\not\in [0,\rho_2]$.
%
%$$
%(pq)^k=\sum_{l=0}^{k}p^{l}q^{k-l}
%$$


\end{document}